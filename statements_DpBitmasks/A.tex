Khi Kefa tới nhà hàng và ngồi vào bàn, anh bồi bàn mang tới cho Kefa bảng menu. Ở đó có n món ăn. Kefa biết rằng anh ta cần ăn chính xác m món. Nhưng anh ta muốn thử nhiều món nhất có thể nên anh ta không muốn đặt một món hai lần.  
\\
\\
Kefa biết rằng món ăn thứ $i$ cho anh ta $a_{i}$ đơn vị của sự thỏa mãn. Một vài món kết hợp với nhau thì rất ngon nhưng một vài món kết hợp lại rất tệ. Do đó, Kefa đặt cho bản thân $k$ luật ăn như sau - nếu anh ta ăn món $x$ ngay trước món $y$ (không có món nào ở giữa món $x$ và món $y$), thì sự thỏa mãn của anh ta tăng thêm $c$.
\\
\\
Hãy giúp Kefa đạt được sự thỏa mãn lớn nhất khi đến nhà hàng!!!.
\\

\textbf{Input}
\\
Dòng đầu tiên chứa ba số nguyên phân tách nhau bởi dấu cách, $n$, $m$ và $k$ $(1 \leq m \leq n \leq 18, 0 \leq k \leq n*(n - 1))$ - Số lượng món ăn trong menu, số lượng món Kefa phải ăn và số lượng luật ăn của Kefa. 
\\

Dòng thứ hai chưa n số nguyên phân $a_{i}$, $(0 \leq a_{i} \leq 10^{9})$ - Sự thỏa mãn từ món ăn thứ $i$.
\\

Tiếp theo là $k$ dòng chưa các luật. Dòng thứ $i$ mô tả bằng ba số nguyên $x_{i}$, $y_{i}$, $c_{i}$ $(1 \leq x_{i}, y_{i} \leq n, 0 \leq c_{i} \leq 10^{9})$. Nghĩa là nếu ăn món $x_{i}$ ngay trước món $y_{i}$, thì sự thỏa mãn của Kefa tăng $c_{i}$. Thỏa mãn rằng, không có cặp chỉ số $i$ và $j$ $(1 \leq i < j \leq k)$, mà $x_{i} = x_{j}$ và $y_{i} = y_{j}$.
\\

\textbf{Output}
\\ 
Một số nguyên duy nhất là mức độ thỏa mãn tối đa mà Kefa có thể đạt được. 
\\

\textbf{Ví dụ}
\begin{table}[h!]
    \begin{center}
        \begin{tabular}{|p{7cm}|p{5cm}|}
            \hline
            \textbf{Input} & \textbf{Output} \\ 
            \hline
            2 2 1 & \\
            1 1 & 3 \\
            2 1 1 & \\
            \hline
            4 3 2 & \\ 
            1 2 3 4 & 12 \\ 
            2 1 5 & \\ 
            3 4 2 & \\
            \hline
        \end{tabular}
    \end{center}
\end{table}

