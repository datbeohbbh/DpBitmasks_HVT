Số Lucifer là số mà hiệu giữa tổng các số ở vị trí chẵn và tổng các số ở vị trí lẻ là số nguyên tố. 
\\
\\
ví dụ số $20314210$ là số lucifer vì:
\begin{itemize}
    \item chữ số tại vị trí lẻ là: 0,2,1,0. 
    \item chữ số tại vị trí chẵn là: 1,4,3,2.
    \item diff = (1 + 4 + 3 + 2) - (0 + 2 + 1 + 0) = 7 - số nguyên tố.
\end{itemize}
Câu hỏi là có bao nhiêu số lucifer từ $l$ tới $r$.
\\

\textbf{Input}
\\
Dòng đầu tiên là số $t$ $(1 \leq t \leq 100)$- số lượng testcases.
\\
$t$ dòng tiếp theo, mỗi dòng chứa hai số $a$ và $b$ $(0 \leq l \leq r \leq 10^{9})$.
\\

\textbf{Output}
\\
In ra $t$ dòng, mỗi dòng là số lượng số lucifer trong đoạn $l$, $r$ cho test tương ứng.
\\

\textbf{Ví dụ}
\begin{table}[h!]
    \begin{center}
        \begin{tabular}{|p{6cm}|p{6cm}|}
            \hline
            \textbf{Input} & \textbf{Output} \\ 
            \hline
            5 & 2 \\ 
            200 250 & 16 \\
            150 200 & 3 \\
            100 150 & 18 \\
            50 100 & 6 \\
            0 50 & \\
            \hline
        \end{tabular}
    \end{center}
\end{table}



